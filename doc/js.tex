JavaScript ist eine clientseitige Script-Sprache für Websiten. Der Name kommt daher, dass die Standard-Obejekte in JavaScript den gleichen Namenskonventionen wie die von Java unterliegen. Ansonsten haben diese zwei Sprachen sehr wenig gemeinsam.

JavaScript ist objektorientiert. Klassen sind ebenfalls Objekte, mit denen prototypen von ihren Instanzen genieriert werden. Prinzipiell sind alle Objekte, Eigenschaften und Methoden Variablen. Die Variablentypisierung ist dynamisch. Es gibt keine Unterscheidung zwischen publiken und privaten Eigenschaften und Methoden.

In Javascript kann auch ohne objektorientierung programmiert werden.