HTML steht für \enquote{Hyper Text Mark-up Language}. Es handelt sich um eine XML-basierte Auszeichnungssprache, welche für den Einsatz auf Websiten gedacht ist.\\
HTML steht in engem Zusammenhang mit HTTP (dem \enquote{Hyper Text Tranfer Protocoll}), welches im Gegensatz zu früheren Netzwerk-Informations-Protokollen (wie zum Beispiel \enquote{Gopher}) den Vorteil, dass es auch Bilder, Videos, Anwendungen und so weiter übertragen kann.

HTML5 ist, wie der Name schon sagt die fünfte Version von HTML. Die für dieses Projekt wichtigste Änderung ist die Einführung des canvas-Elements.