In diesem Projekt wurden zwei verschiedene Sensoren verwendet: der Kompass und der Beschleunigungssensor.\\
\\
Der Bewegungssensor funktioniert mittels eines beweglichen Ankers und mindestens 4 festsitzenden Kondensatorelektroden. Wenn sich nun der Anker bewegt, ändert sich der Abstand zu den Elektrode. Dies führt zu Änderungen der Kapazitäten, welche gemessen werden können.
\\
\\
Der Kompass misst das ihn umgebende elektrische Feld und errechnet sich daraus die Richtung, in welcher 
Norden liegt. Durch andere elektrische Geräte oder Magneten in der Nähe kann das Erdmagnetfeld jedoch überlagert werden, wodurch der Kompass nicht mehr nach Norden zeigt.

